%%%%%%%%%%%%%%%%%%%%%%%%%%%%%%%%%%%%%%%%%%%%%%%%%%%%%%%%%%%%%%%%%%
% PREAMBOLO: Configurazione basata su unina_doc_class
%%%%%%%%%%%%%%%%%%%%%%%%%%%%%%%%%%%%%%%%%%%%%%%%%%%%%%%%%%%%%%%%%%
\documentclass{unina_doc_class}

% --- FIX PER WINDOWS: Sovrascriviamo i comandi Unix-only ---
% Questi comandi nel .cls usano 'rm' e 'date' di Linux che rompono la compilazione su Windows.
% Li sostituiamo con testo statico.
\renewcommand{\gitparse}{N/A}
\renewcommand{\epoch}{\today} 
% -----------------------------------------------------------

% --- Definizioni Variabili Globali per la Copertina ---
\gdef\documentname{Note di Algebra Lineare e Geometria}
\gdef\documentyear{Un'introduzione concettuale}
\gdef\xauthors{\hi{\iuser Rielaborazione da Appunti}} 
\gdef\yauthors{} 
\gdef\zauthors{} 

% Titoli per i listing di codice
\gdef\documentlsttitle{Frammenti di Codice}
\gdef\documentlstlabel{Codice}

% --- Pacchetti aggiuntivi ---
\usepackage[italian]{babel}

% --- Definizione Stili Personalizzati per i Box ---
\newtcolorbox{boxesempio}{
    enhanced,
    colback=azure-gradient-1!20!white,
    colframe=primary,
    fonttitle=\bfseries,
    title=Esempio,
    breakable,
    drop fuzzy shadow,
    attach boxed title to top left={xshift=5mm,yshift*=-\tcboxedtitleheight/2},
    boxed title style={colback=primary}
}

\newtcolorbox{boxesercizio}[1][Esercizio Prototipo]{
    enhanced,
    colback=green-gradient-1!20!white,
    colframe=green!60!black,
    fonttitle=\bfseries,
    title={#1},
    breakable,
    drop fuzzy shadow,
    attach boxed title to top left={xshift=5mm,yshift*=-\tcboxedtitleheight/2},
    boxed title style={colback=green!60!black}
}

% --- Ambienti Matematici ---
\theoremstyle{definition}
\newtheorem{definizione}{Definizione}[chapter]
\newtheorem{teorema}[definizione]{Teorema}
\newtheorem{osservazione}[definizione]{Osservazione}
\newtheorem{proposizione}[definizione]{Proposizione}

% --- Scorciatoie Matematiche ---
\newcommand{\K}{\mathbb{K}}
\newcommand{\R}{\mathbb{R}}
\newcommand{\C}{\mathbb{C}}
\newcommand{\Span}{\text{Span}}
\newcommand{\Ker}{\text{Ker}}
\newcommand{\Img}{\text{Im}}
\newcommand{\dimn}{\text{dim}}
\newcommand{\rk}{\text{rk}}
\newcommand{\vl}{\text{VL}}
\newcommand{\vd}{\text{VD}}
\newcommand{\Hom}{\text{Hom}}
\newcommand{\Giac}{\text{Giac}}

\pagestyle{plain}

%%%%%%%%%%%%%%%%%%%%%%%%%%%%%%%%%%%%%%%%%%%%%%%%%%%%%%%%%%%%%%%%%%
% CORPO DEL DOCUMENTO
%%%%%%%%%%%%%%%%%%%%%%%%%%%%%%%%%%%%%%%%%%%%%%%%%%%%%%%%%%%%%%%%%%
\begin{document}

 % Genera la copertina
 \coverpage{26}{\documentyear}{30}{\documentname}

 % Introduzione
 \introduction{Nome Docente}{Queste note sono una rielaborazione personale.}

 \tableofcontents
 \newpage

 \chapter{Il Problema di Partenza: Sistemi Lineari}

 L'algebra lineare, nella sua essenza, nasce da un problema molto pratico: trovare un modo sistematico ed efficiente per risolvere insiemi di equazioni, specialmente quando le equazioni e le incognite diventano numerose.

 \section{Equazioni e Sistemi}
 Iniziamo definendo il nostro oggetto di studio fondamentale.

 \begin{definizione}[Sistema Lineare]
 Un \textbf{sistema lineare} $m \times n$ (leggi: "m per n") è un insieme di $m$ equazioni lineari in $n$ incognite (o variabili) $x_1, \dots, x_n$.
 La sua forma canonica è:
 $$
 \begin{cases}
     a_{11}x_1 + a_{12}x_2 + \dots + a_{1n}x_n = b_1 \\
     a_{21}x_1 + a_{22}x_2 + \dots + a_{2n}x_n = b_2 \\
     \vdots \\
     a_{m1}x_1 + a_{m2}x_2 + \dots + a_{mn}x_n = b_m
 \end{cases}
 $$
 dove i numeri $a_{ij}$ sono i \textbf{coefficienti} e i $b_i$ sono i \textbf{termini noti}.
 \end{definizione}

 \paragraph{A cosa serve questa definizione?}
 Stabilisce il nostro campo di gioco. Qualsiasi problema lineare, dalla fisica all'economia, può essere modellato in questa forma.

 \begin{boxesempio}
 \begin{enumerate}
     \item \textbf{Sistema $2 \times 2$:} Due equazioni in due incognite.
     $$ \begin{cases} x_1 + x_2 = 3 \\ 2x_1 - x_2 = 0 \end{cases} $$
    
     \item \textbf{Sistema $2 \times 3$:} Due equazioni in tre incognite.
     $$ \begin{cases} x + y + z = 1 \\ 2x - y = 0 \end{cases} $$
 \end{enumerate}
 \end{boxesempio}

 \paragraph{La Forma della Soluzione: Vettori e n-uple}
 Quando risolviamo un sistema con $n$ incognite, il risultato non è un numero singolo, ma una lista ordinata di $n$ valori (uno per ogni variabile). In Algebra Lineare, chiamiamo questa lista \textbf{vettore} o \textbf{$n$-upla}.

 Possiamo rappresentare la soluzione in due modi grafici equivalenti:
 \begin{itemize}
     \item \textbf{In orizzontale (n-upla):} $v = (2, \ 4, \ -3, \ 1)$
     \item \textbf{In verticale (Vettore colonna):} $v = \begin{pmatrix} 2 \\ 4 \\ -3 \\ 1 \end{pmatrix}$
 \end{itemize}
 Sebbene il significato sia lo stesso, nel contesto delle matrici e per scrivere l'insieme delle soluzioni $V(S)$, useremo quasi sempre la \textbf{notazione in colonna}.

 \begin{definizione}[Soluzione e Compatibilità]
 Una \textbf{soluzione} del sistema è una $n$-upla di numeri $(\alpha_1, \dots, \alpha_n)$ che, sostituiti alle incognite $(x_1, \dots, x_n)$, rendono vere \emph{tutte} le equazioni simultaneamente. L'insieme di tutte le soluzioni si chiama $V(S)$.
 \begin{itemize}
     \item Se $V(S) \neq \emptyset$ (esiste almeno una soluzione), il sistema è \textbf{compatibile}.
     \item Se $V(S) = \emptyset$ (non esistono soluzioni), il sistema è \textbf{incompatibile}.
 \end{itemize}
 \end{definizione}

 \paragraph{A cosa serve?}
 Questa definizione classifica i sistemi in due categorie fondamentali: quelli che hanno una risposta (compatibili) e quelli che non ce l'hanno (incompatibili).

 \begin{boxesempio}
 \begin{enumerate}
     \item \textbf{Compatibile (Soluzione Unica):}
     $ \begin{cases} x = 5 \\ y = 2 \end{cases} $ \\ 
     L'unica soluzione è $V(S) = \{(5, 2)\}$.
 \end{enumerate}
 \end{boxesempio}

 Vedremo più tardi che le soluzioni non saranno sempre così semplici da rappresentare.

 \begin{definizione}[Sistema Omogeneo]
 Un sistema lineare si dice \textbf{omogeneo} se tutti i termini noti sono nulli ($b_1 = \dots = b_m = 0$).
 \end{definizione}

 \chapter{Il Metodo Risolutivo: Matrici e Algoritmo di Gauss}

 Risolvere un sistema "a occhio" o per sostituzione diventa impossibile appena le dimensioni crescono. Abbiamo bisogno di un metodo sistematico.
 Per farlo, per prima cosa dobbiamo liberarci della notazione ingombrante ($x, y, z, =, \dots$) e concentrarci solo sui numeri.

 \section{Il Linguaggio Compatto: Le Matrici}

 Notiamo che in un sistema lineare, le incognite sono solo dei "segnaposto" ordinati. Ciò che determina la soluzione sono i coefficienti.

 \begin{definizione}[Matrice Associata a un Sistema]
 Una \textbf{matrice} è una tabella rettangolare di numeri. Dato un sistema lineare, possiamo estrarre due matrici fondamentali:
 \begin{itemize}
     \item \textbf{Matrice Incompleta ($A$):} Contiene solo i coefficienti delle incognite.
     \item \textbf{Matrice Completa o Aumentata ($(A|b)$):} È la matrice $A$ con l'aggiunta di una colonna finale separata (spesso da una linea verticale) che contiene i termini noti $b$.
 \end{itemize}
 \end{definizione}

 \paragraph{A cosa serve?}
 Un "sistema lineare" $S$ \emph{è} la sua matrice completa $(A|b)$. Lavorare sulla matrice è più pulito, più veloce e riduce gli errori di distrazione.

 \begin{boxesempio}
 \textbf{Dal Sistema alla Matrice:}
 $$ S: \begin{cases} x - 2y + z = 0 \\ 3x + y = 5 \\ z = 2 \end{cases} \implies (A|b) = \left(\begin{array}{ccc|c} 
 1 & -2 & 1 & 0 \\ 
 3 & 1 & 0 & 5 \\ 
 0 & 0 & 1 & 2 
 \end{array}\right) $$
 Notare lo $0$ inserito dove manca l'incognita (nella seconda e terza equazione).
 \end{boxesempio}

 \section{L'Algoritmo di Gauss sulle Matrici}

 Ora che abbiamo la matrice, l'obiettivo è semplificarla. L'Algoritmo di Gauss usa delle "mosse" che non cambiano le soluzioni del sistema, ma trasformano la matrice in una forma facile da leggere.
 In particolare per risolvere un sistema lineare abbiamo 2 fasi. Il 1° è quello di portare la matrice in \textbf{Forma a Scala} la 2° e applicare l'algoritmo di Gauss.

 \begin{definizione}[Mosse di Riga]
 Le seguenti operazioni sulle righe della matrice producono una matrice \textbf{equivalente} (corrispondente a un sistema con le stesse soluzioni):
 \begin{enumerate}
     \item \textbf{Scambio ($R_{i \leftrightarrow j}$):} Scambiare la riga $i$ con la riga $j$.
     \item \textbf{Moltiplicazione ($R_i \leftarrow \lambda R_i$):} Moltiplicare tutti gli elementi della riga $i$ per uno scalare $\lambda \neq 0$.
     \item \textbf{Combinazione ($R_i \leftarrow R_i + \lambda R_j$):} Sostituire la riga $i$ con la somma di se stessa e un multiplo della riga $j$.
 \end{enumerate}
 \end{definizione}

 L'obiettivo di queste mosse è raggiungere la \textbf{Forma a Scala}.

 \begin{definizione}[Forma a Scala e Pivot]
 Una matrice è detta \textbf{a scala} se soddisfa una precisa condizione visiva basata sulle colonne:
 \begin{enumerate}
     \item Il primo numero non nullo di ogni riga è chiamato \textbf{Pivot}.
     \item Ogni Pivot deve trovarsi rigorosamente \textbf{più a destra} del Pivot della riga precedente.
     \item Sotto ogni Pivot, tutti gli elementi della colonna devono essere \textbf{zero}.
 \end{enumerate}
 \end{definizione}

 \paragraph{Interpretazione Pratica: La Regola delle Colonne}
 Guardando la matrice per colonne da sinistra a destra:
 \begin{itemize}
     \item Appena troviamo un Pivot in una colonna, quella colonna è "conquistata" da quella riga.
     \item Nessun'altra riga successiva può iniziare in quella colonna.
     \item La matrice deve assumere una forma a "triangolo di zeri" nell'angolo in basso a sinistra.
 \end{itemize}

 \begin{boxesempio}
 \textbf{Esempio di Riduzione $4 \times 4$:}
 Consideriamo la matrice completa associata a un sistema di 4 equazioni in 4 incognite:
 $$ 
 \left(\begin{array}{cccc|c} 
 \underline{1} & 1 & 1 & 1 & 4 \\ 
 2 & 3 & 4 & 1 & 5 \\ 
 0 & 2 & 2 & 0 & 2 \\ 
 1 & 2 & 2 & 2 & 6 
 \end{array}\right) 
 $$

 \textbf{Passo 1: Sistemare la Colonna 1}
 \begin{itemize}
     \item \textbf{Analisi:} Il pivot è l'1 in alto a sinistra ($a_{11}$). Sotto di esso abbiamo un $2$, uno $0$ e un $1$. Lo $0$ va bene, ma il $2$ e l'1 vanno eliminati.
     \item \textbf{Azione:} 
     \begin{enumerate}
         \item $R_2 \leftarrow R_2 - 2R_1$ (per eliminare il 2)
         \item $R_4 \leftarrow R_4 - R_1$ (per eliminare l'1)
     \end{enumerate}
 \end{itemize}
 $$ 
 \left(\begin{array}{cccc|c} 
 \underline{1} & 1 & 1 & 1 & 4 \\ 
 0 & \underline{1} & 2 & -1 & -3 \\ 
 0 & 2 & 2 & 0 & 2 \\ 
 0 & 1 & 1 & 1 & 2 
 \end{array}\right) 
 $$

 \textbf{Passo 2: Sistemare la Colonna 2}
 \begin{itemize}
     \item \textbf{Analisi:} Ora ignoriamo la prima riga. Il nuovo "capo" è il pivot nella riga 2, colonna 2 (valore $\underline{1}$). Sotto di esso c'è un $2$ e un $1$. Dobbiamo fare pulizia.
     \item \textbf{Azione:}
     \begin{enumerate}
         \item $R_3 \leftarrow R_3 - 2R_2$
         \item $R_4 \leftarrow R_4 - R_2$
     \end{enumerate}
 \end{itemize}
 $$ 
 \left(\begin{array}{cccc|c} 
 \underline{1} & 1 & 1 & 1 & 4 \\ 
 0 & \underline{1} & 2 & -1 & -3 \\ 
 0 & 0 & -2 & 2 & 8 \\ 
 0 & 0 & -1 & 2 & 5 
 \end{array}\right) 
 $$

 \textbf{Passo 3: Sistemare la Colonna 3}
 \begin{itemize}
     \item \textbf{Analisi:} Siamo alla riga 3. Il candidato pivot è $-2$. Sotto di esso c'è un $-1$.
     \item \textbf{Trucco (Scambio):} Notiamo che $-1$ è aritmeticamente più semplice di $-2$. Scambiamo le righe 3 e 4 per comodità ($R_3 \leftrightarrow R_4$), portando il $-1$ in posizione di pivot.
 \end{itemize}
 $$ 
 \left(\begin{array}{cccc|c} 
 \underline{1} & 1 & 1 & 1 & 4 \\ 
 0 & \underline{1} & 2 & -1 & -3 \\ 
 0 & 0 & \underline{-1} & 2 & 5 \\ 
 0 & 0 & -2 & 2 & 8 
 \end{array}\right) 
 $$
 \begin{itemize}
     \item \textbf{Azione Finale:} Ora usiamo il pivot $-1$ per eliminare il $-2$ sotto di esso: $R_4 \leftarrow R_4 - 2R_3$.
 \end{itemize}
 $$ 
 \left(\begin{array}{cccc|c} 
 \underline{1} & 1 & 1 & 1 & 4 \\ 
 0 & \underline{1} & 2 & -1 & -3 \\ 
 0 & 0 & \underline{-1} & 2 & 5 \\ 
 0 & 0 & 0 & \underline{-2} & -2 
 \end{array}\right) 
 $$

 \textbf{Verifica Finale:}
 Ogni riga inizia con un pivot (sottolineato) che è più a destra del precedente. Sotto ogni pivot ci sono solo zeri. La matrice è a scala.
 \end{boxesempio}

 \paragraph{Classificazione delle Variabili}
 \begin{itemize}
     \item \textbf{Variabili Dipendenti ($\vd$):} Sono le incognite che corrispondono alle colonne contenenti i pivot.
     \item \textbf{Variabili Libere ($\vl$):} Sono le incognite che corrispondono alle colonne \emph{senza} pivot. Queste diventeranno i nostri parametri.
 \end{itemize}

 \section{Algoritmo di Gauss}

 Ora che il sistema lineare è scala non ci resta altro che applicare l'algoritmo di Gauss.
 La 1° cosa da fare è riportare la matrice in sistema lineare. E poi eseguire i seguenti passi:

 \begin{itemize}
     \item \textbf{Controllo Compatibilità:} Se notiamo una riga del tipo $(0 \dots 0 \ | \ k)$ con $k \neq 0$ (cioè un pivot cade nell'ultima colonna, quella dei termini noti), il sistema è \textbf{impossibile}. In tal caso $V(S) = \emptyset$.
     \item Partendo dal basso verso l'alto, portiamo tutti i termini NON pivot a destra dell'uguale.
     \item Sostituiamo il termine di pivot all'equazioni che sono sopra la riga designata.
     \item Ripetere il processo fino ad arrivare alla 1° riga.
 \end{itemize}

 \textbf{Nota bene:} Il numero di variabili libere determina la "dimensione" della soluzione (0 = soluzione unica, 1 = retta di soluzioni, ecc.).

 \begin{boxesempio}
 \textbf{Conclusione Esempio $4 \times 4$:}
 Riprendiamo la matrice ridotta a scala ottenuta nel passo precedente:
 $$ 
 \left(\begin{array}{cccc|c} 
 \underline{1} & 1 & 1 & 1 & 4 \\ 
 0 & \underline{1} & 2 & -1 & -3 \\ 
 0 & 0 & \underline{-1} & 2 & 5 \\ 
 0 & 0 & 0 & \underline{-2} & -2 
 \end{array}\right) 
 $$

 \begin{itemize}
     \item \textbf{Analisi Compatibilità:} L'ultimo pivot ($-2$) è sulla quarta colonna (coefficiente di $x_4$), e \emph{non} sulla colonna dei termini noti. Il sistema è \textbf{compatibile}.
     \item \textbf{Analisi Variabili:}
     \begin{itemize}
         \item Ci sono pivot nelle colonne 1, 2, 3 e 4. Quindi $\vd = \{x_1, x_2, x_3, x_4\}$.
         \item Non ci sono colonne senza pivot. Quindi $\vl = \emptyset$.
     \end{itemize}
     Poiché il numero di variabili libere è 0, la soluzione esiste ed è \textbf{unica}.
 \end{itemize}

 \textbf{Sostituzione all'indietro:}
 Riscriviamo il sistema partendo dall'ultima equazione e risalendo:
 $$ 
 \begin{cases} 
 x_1 + x_2 + x_3 + x_4 = 4 \\ 
 x_2 + 2x_3 - x_4 = -3 \\ 
 -x_3 + 2x_4 = 5 \\ 
 -2x_4 = -2 
 \end{cases} 
 $$

 Calcoliamo i valori dal basso verso l'alto:
 \begin{enumerate}
     \item Dalla 4\textsuperscript{a} eq: $-2x_4 = -2 \implies x_4 = \mathbf{1}$.
     \item Sostituisco in 3\textsuperscript{a}: $-x_3 + 2(1) = 5 \implies -x_3 = 3 \implies x_3 = \mathbf{-3}$.
     \item Sostituisco in 2\textsuperscript{a}: $x_2 + 2(-3) - (1) = -3 \implies x_2 - 7 = -3 \implies x_2 = \mathbf{4}$.
     \item Sostituisco in 1\textsuperscript{a}: $x_1 + 4 + (-3) + 1 = 4 \implies x_1 + 2 = 4 \implies x_1 = \mathbf{2}$.
     \item Per scrivere la soluzione finale, partiamo col creare un vettore colonna con i valori trovati dall'alto verso il basso.
 \end{enumerate}

 \textbf{Scrittura finale della soluzione:}
 $$ V(S) = \left\{ \begin{pmatrix} 4 \\ -3 \\ 5 \\ 2 \end{pmatrix} \right\} $$
 \end{boxesempio}

 \chapter{La Struttura Astratta: Spazi Vettoriali}

 \section{Definizioni e Sottospazi}

 \begin{definizione}[K-Spazio Vettoriale]
 Un \textbf{K-Spazio Vettoriale} (o spazio vettoriale sul campo $\K$) è un insieme non vuoto $V$ dotato di due operazioni:
 \begin{enumerate}
     \item \textbf{Somma Interna} ($+$): $V \times V \to V$, $(v_1, v_2) \mapsto v_1 + v_2$
     \item \textbf{Prodotto per Scalare} ($\cdot$): $\K \times V \to V$, $(\lambda, v) \mapsto \lambda v$
 \end{enumerate}
 Queste operazioni devono soddisfare 8 assiomi.
 \end{definizione}

 \paragraph{A cosa serve?}
 In poche parole ci dice che qualsiasi altro "oggeto matematico" che rispetta quelle caratteristiche può essere ricondotto a vettore.

 \begin{boxesempio}
 \begin{enumerate}
     \item \textbf{Vettori}: Già noti.
     \item \textbf{$M_{2,2}(\R)$ (Matrici):} L'insieme delle matrici $2 \times 2$.
     $$ \begin{pmatrix} a & b \\ c & d \end{pmatrix} + \begin{pmatrix} a' & b' \\ c' & d' \end{pmatrix} = \begin{pmatrix} a+a' & b+b' \\ c+c' & d+d' \end{pmatrix} $$
     \item \textbf{$\R_{\le 2}[T]$ (Polinomi):} L'insieme dei polinomi di grado $\le 2$, $P(T) = aT^2+bT+c$.
 \end{enumerate}
 \end{boxesempio}

 \begin{definizione}[Sottospazio Vettoriale]
 Un sottoinsieme non vuoto $W \subseteq V$ è un \textbf{sottospazio vettoriale} se è esso stesso uno spazio vettoriale (usando le stesse operazioni di $V$).
 \end{definizione}

 \begin{proposizione}[Test del Sottospazio]
 $W \subseteq V$ (con $W \neq \emptyset$) è un sottospazio se e solo se:
 \begin{enumerate}
     \item $\forall w_1, w_2 \in W \implies w_1 + w_2 \in W$ (Chiuso per la somma).
     \item $\forall \lambda \in \K, \forall w \in W \implies \lambda w \in W$ (Chiuso per il prodotto scalare).
 \end{enumerate}
 \end{proposizione}

 \section{Come trattare Polinomi e Matrici come Vettori}

 Finora abbiamo applicato l'Algoritmo di Gauss a vettori colonna standard. Ma l'Algebra Lineare si applica anche a oggetti più complessi, come polinomi o intere matrici. Possiamo "tradurre" qualsiasi spazio vettoriale di dimensione finita in vettori colonna tramite il concetto di \textbf{Coordinate}.

 \subsection{Caso 1: Dai Polinomi ai Vettori}
 Un polinomio è definito dai suoi coefficienti. La variabile $t$ (o $x$) è solo un segnaposto per indicare la posizione.
 $$ P(t) = a_0 + a_1 t + a_2 t^2 + \dots + a_n t^n \quad \longleftrightarrow \quad v_P = \begin{pmatrix} a_0 \\ a_1 \\ a_2 \\ \vdots \\ a_n \end{pmatrix} $$

 \begin{boxesempio}
 \textbf{Traduzione di Polinomi}
 Siamo nello spazio dei polinomi di grado $\le 2$.
 \begin{itemize}
     \item $P_1(t) = 3 - 5t + t^2 \implies v_1 = \begin{pmatrix} 3 \\ -5 \\ 1 \end{pmatrix}$
     \item $P_2(t) = t^2 - 4 \implies v_2 = \begin{pmatrix} -4 \\ 0 \\ 1 \end{pmatrix}$ (Attenzione al termine $t$ mancante!)
     \item $P_3(t) = 2t \implies v_3 = \begin{pmatrix} 0 \\ 2 \\ 0 \end{pmatrix}$
 \end{itemize}
 \end{boxesempio}

 \chapter{Descrivere Spazi Vettoriali con Basi e Dimensione}

 \section{Generatori e Rappresentazioni Parametriche}

 \begin{definizione}[Combinazione Lineare e Span]
 Dato un insieme di vettori $U = \{v_1, \dots, v_s\}$ in $V$.
 \begin{itemize}
     \item Una \textbf{combinazione lineare} è un qualsiasi vettore $v$ della forma:
     $$ v = \alpha_1 v_1 + \dots + \alpha_s v_s, \quad \text{con } \alpha_i \in \K $$
     \item Lo \textbf{Span} di $U$, $\Span(U)$, è l'insieme di \emph{tutte} le possibili combinazioni lineari di $U$.
 \end{itemize}
 Lo $\Span(U)$ è il più piccolo sottospazio vettoriale di $V$ che contiene $U$. $U$ è un \textbf{insieme di generatori} per $\Span(U)$.
 \end{definizione}

 \section{Indipendenza Lineare}

 \begin{definizione}[Indipendenza Lineare]
 Un insieme di vettori $U = \{v_1, \dots, v_s\}$ si dice \textbf{linearmente indipendente} se nessun vettore è combinazione lineare degli altri.
 Formalmente: l'unica soluzione dell'equazione omogenea $\alpha_1 v_1 + \dots + \alpha_s v_s = O_V$ è la soluzione \textbf{banale} $\alpha_1 = \dots = \alpha_s = 0$.
 \end{definizione}

 \section{Basi e Dimensione}

 \begin{definizione}[Base]
 Una \textbf{Base} $\mathcal{B} = \{v_1, \dots, v_n\}$ di uno spazio vettoriale $V$ è un insieme di vettori che soddisfa due condizioni:
 \begin{enumerate}
     \item $\Span(\mathcal{B}) = V$ (Sono generatori di tutto lo spazio).
     \item $\mathcal{B}$ è linearmente indipendente (Non sono ridondanti).
 \end{enumerate}
 \end{definizione}

 \begin{teorema}[Invarianza della Dimensione]
 Tutte le basi di $V$ hanno lo stesso numero di elementi. Questo numero si chiama \textbf{Dimensione} di $V$, $\dimn(V)$.
 \end{teorema}

 \begin{boxesempio}
 \begin{enumerate}
     \item $\dimn(\R^3) = 3$
     \item $\dimn(\R_{\le 2}[T]) = 3$
     \item $\dimn(M_{2,2}(\R)) = 4$
 \end{enumerate}
 \end{boxesempio}

 \chapter{Trasformare gli Spazi: Applicazioni Lineari}

 \section{Definizione e Proprietà}

 \begin{definizione}[Applicazione Lineare]
 Un'applicazione $T: V \to W$ tra due K-spazi vettoriali $V$ e $W$ è \textbf{lineare} se conserva le operazioni:
 \begin{enumerate}
     \item \textbf{Additività:} $T(v_1 + v_2) = T(v_1) + T(v_2)$
     \item \textbf{Omogeneità:} $T(\lambda v) = \lambda T(v)$
 \end{enumerate}
 \end{definizione}

 \begin{osservazione}
 Una condizione necessaria per la linearità è che l'origine sia mappata nell'origine: $T(O_V) = O_W$.
 \end{osservazione}

 \section{Nucleo e Immagine}

 \begin{definizione}[Nucleo (Kernel)]
 Il \textbf{Nucleo} di $T$ ($\Ker(T)$) è l'insieme dei vettori del dominio $V$ che vengono mappati nel vettore nullo del codominio $W$:
 $$ \Ker(T) = \{v \in V \mid T(v) = O_W\} $$
 \end{definizione}

 \begin{definizione}[Immagine (Image)]
 L'\textbf{Immagine} di $T$ ($\Img(T)$) è l'insieme dei vettori del codominio $W$ che sono "raggiunti" dalla trasformazione:
 $$ \Img(T) = \{w \in W \mid \exists v \in V \text{ t.c. } T(v) = w\} $$
 \end{definizione}

 \begin{teorema}[Rango-Nullità]
 $$ \dimn(V) = \dimn(\Ker(T)) + \dimn(\Img(T)) $$
 \end{teorema}

 \begin{boxesercizio}[Calcolo di Basi per Nucleo e Immagine]
 Consideriamo $T: \R^3 \to \R^3$ definita da $T\begin{pmatrix} x \\ y \\ z \end{pmatrix} = \begin{pmatrix} x-y+2z \\ 3x-y \\ 2x-y+z \end{pmatrix}$.
 Trovare una base di $\Ker(T)$ e $\Img(T)$.

 \textbf{Svolgimento:}
 \begin{enumerate}
     \item \textbf{Matrice Associata $A$:}
     $$ A = \begin{pmatrix} 1 & -1 & 2 \\ 3 & -1 & 0 \\ 2 & -1 & 1 \end{pmatrix} $$
    
     \item \textbf{Base del Nucleo $\Ker(T)$:}
     Risolviamo $A \mathbf{v} = \mathbf{0}$. Riduciamo $A$ a scala:
     $$ A \sim \begin{pmatrix} 1 & -1 & 2 \\ 0 & 1 & -3 \\ 0 & 0 & 0 \end{pmatrix} $$
     Variabile libera $z$. Base $\Ker(T) = \left\{ \begin{pmatrix} 1 \\ 3 \\ 1 \end{pmatrix} \right\}$.

     \item \textbf{Base dell'Immagine $\Img(T)$:}
     Corrisponde allo Span delle colonne pivot della matrice originale $A$ (colonne 1 e 2).
     Base $\Img(T) = \left\{ \begin{pmatrix} 1 \\ 3 \\ 2 \end{pmatrix}, \begin{pmatrix} -1 \\ -1 \\ -1 \end{pmatrix} \right\}$.
 \end{enumerate}
 \end{boxesercizio}

\end{document}